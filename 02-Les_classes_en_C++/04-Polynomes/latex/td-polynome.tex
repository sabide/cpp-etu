\documentclass[a4paper,14pt]{article}
\usepackage[utf8]{inputenc}   % Encodage des caractères du fichier source
\usepackage[T1]{fontenc}      % Encodage des caractères pour les polices
\usepackage[french]{babel}    % Optionnel : support pour la typographie française

\usepackage{amsmath}
\usepackage{amssymb}
\usepackage{graphicx}
\usepackage{listings}
\usepackage{hyperref}
\usepackage{geometry}
\usepackage{xcolor}

\geometry{margin=1in}

\lstset{
    language=C++,
    basicstyle=\ttfamily\small,
    keywordstyle=\color{blue}\bfseries,
    stringstyle=\color{red},
    commentstyle=\color{green},
    numbers=left,
    numberstyle=\tiny\color{gray},
    stepnumber=1,
    numbersep=10pt,
    backgroundcolor=\color{white},
    tabsize=4,
    showspaces=false,
    showstringspaces=false,
    breaklines=true
}


\title{TD 6 : Définition de classes, surcharge d'opérateurs}
\author{}
\date{}

\begin{document}

% Ligne horizontale supérieure
\hrule height 1pt
\vspace{0.1cm}
% Titre du document
\noindent
\makebox[\textwidth]{{\small Univ. Côte d'Azur} \hfill \textbf{EPU-MAM4} }
\makebox[\textwidth]{{\small 2024-2025} \hfill \textbf{Programmation C++}}
\begin{center}
{\Large \bf Définition de classes, surcharge d'opérateurs}
\end{center}
\hrule height 1pt
\vspace{0.5cm}
% Ligne horizontale inférieure

%% https://perso.lpsm.paris/~rouxr/enseignements/1415/C_Cpp/

\noindent L'objectif est de créer une classe Polynome afin de réaliser des calculs formels sur les polynômes (d'une variable réelle).
Voici ce que l'on aimerait faire avec la classe Polynôme :

\begin{lstlisting}[language=C++]
#include <iostream>
#include "polynome.hpp"

using namespace std;

int main(void) {
    double coeff[] = {2, 4, -1, 0, 2};
    unsigned degre[] = {0, 2, 3, 5, 6}; // on suppose le tableau ordonne
    
    Polynome P(coeff, degre, 5);
    cout << "P:\t" << P << endl;

    Polynome Q(4, 2);
    cout << "-Q:\t" << -Q << endl;

    cout << "1+Q:\t" << 1+Q << endl;
    cout << "P+Q:\t" << P+Q << endl;

    cout << "Evaluation de P-Q en 3.14:\t" << (P-Q)(3.14) << endl;
    return 0;
}
\end{lstlisting}

\noindent Ce programme doit donner le résultat suivant :
\begin{verbatim}
    P: 2 + 4x^2 + -1x^3 + 2x^6
    -Q: -4x^2
    1+Q: 1 + 4x^2
    P+Q: 2 + 8x^2 + -1x^3 + 2x^6
    Evaluation de P-Q en 3.14: 1887.98
\end{verbatim}

\section*{Classe \texttt{Monome}}

On débute par créer la classe « Monome » qui code des monômes de la forme $ax^d$ où $a$ est le coefficient réel et $d$ le degré. La déclaration de cette classe doit se faire dans le fichier « polynome.hpp ».
\begin{lstlisting}[language=C++]
class Monome {
public:
    Monome(double coeff, unsigned degre);
    unsigned degre() const;
    double coeff() const;
private:
    double c;
    unsigned d;
};
\end{lstlisting}

\begin{enumerate}
    \item Ajoutez les arguments par défaut \texttt{coeff = 0} et \texttt{degre = 0} au constructeur de la classe \texttt{Monome}. Définir dans la classe le constructeur et les méthodes constantes \texttt{degre()} (qui renvoie le degré du monôme) et coeff() (qui renvoie le coefficient c). Ces méthodes seront donc \texttt{inline}.
    
    \item Testez différents appels du constructeur pour définir des monômes. Le constructeur par défaut a-t-il été défini ? Comment doit-on l'appeler ? Est-il nécessaire de définir le constructeur de clonage ? Le destructeur ? Si oui, faites-le.
\end{enumerate}

\section*{Classe \texttt{Polynome}}

On définit la classe \texttt{Polynome} dans le fichier \texttt{polynome.hpp}.
Dans notre modèle on représente un polynôme de degré $n$ par un tableau dynamique de $n+1$ monômes.

\begin{enumerate}
    \setcounter{enumi}{2}
    \item Définir la classe \texttt{Polynome} qui contient 2 champs privés : \texttt{n} qui code le degré du polynôme et \texttt{data} destiné à contenir les $n+1$ monômes. Ajouter la méthode publique \texttt{degre()} similaire à celle de la classe \texttt{Monome}.
    
    \item Définir le constructeur utilisé à la ligne 10 du programme de test qui servira aussi de constructeur par défaut. La définition se fera dans un fichier \texttt{polynome.cpp}.
    
    \item Déclarer dans la classe et définir dans le fichier \texttt{polynome.cpp} les constructeurs suivants :
    \begin{itemize}
        \item le constructeur de la ligne 8,
        \item le constructeur de clonage/copie,
        \item le constructeur qui prend une référence constante sur un \texttt{Monome} (permettra de convertir un \texttt{Monome} en \texttt{Polynome}).
    \end{itemize}
    
    \item Est-il nécessaire de définir le constructeur de clonage ? Le destructeur ? Si oui, faites-le.
\end{enumerate}

\section*{Surcharge des opérateurs (par des méthodes)}

Commençons par surcharger l'opérateur d'affectation \texttt{=} de la classe \texttt{Polynome}. Si cet opérateur n'est pas défini par le programmeur, il est généré par le systéme, mais cette version peut être erronée (c'est le cas pour la classe \texttt{Polynome} mais pas pour la classe \texttt{Monome}).

\begin{enumerate}
    \setcounter{enumi}{6}
    \item Définir la méthode de nom \texttt{operator=} en complétant le code suivant.
    
    \begin{lstlisting}[language=C++]
    Polynome& Polynome::operator=(Polynome const& P) {
        if (&P == this) return *this;

        // sinon: recopie du polymoe P dans l'objet courant
        // si necessaire : detruire data puis reallouer
    };
    \end{lstlisting}

    \item Définir la méthode de nom \texttt{operator()} et de prototype \texttt{double operator()(double x) const} dans les 2 classes \texttt{Monome} et \texttt{Polynome}.
    
    \item Définir la méthode \texttt{operator-} de prototype \texttt{Polynome operator-() const} qui surcharge l'opérateur unaire \texttt{-} utilisé ligne 11.
\end{enumerate}

\section*{Surcharge des opérateurs (par des fonctions globales)}

La surcharge de l'opérateur de flux \texttt{<<} doit se faire par une fonction globale de prototype \verb|std::ostream\& operator<<(std::ostream\& o, Polynome const\& P)|. Cette fonction doit renvoyer une référence sur le flux \verb|o| modifié par l'affichage de \verb|P|.

\begin{enumerate}
    \setcounter{enumi}{9}
    \item écrire le prototype de cette fonction précédé du mot-clé \texttt{friend} dans la classe \texttt{Polynome} (cela permet à cette fonction d'accéder au champ privé \texttt{data}). Définir cette fonction (globale et amie de la classe \texttt{Polynome}) dans le fichier \verb|polynome.cpp|. L'affichage doit s'inspirer de celui donné en exemple.
    
    \item De la même façon que pour l'opérateur \texttt{<<}, surchargez les opérateurs arithmétiques binaires \texttt{+}, \texttt{-} et \texttt{*} par des fonctions globales de nom \texttt{operator+}, \texttt{operator-} et \texttt{operator*}. Le prototype de la fonction qui surcharge l'opérateur \texttt{+} entre 2 \texttt{Polynome} est par exemple :
    
    \begin{lstlisting}[language=C++]
    Polynome operator+(Polynome const& P, Polynome const& Q);
    \end{lstlisting}
    
    \item Déclarer ces fonctions comme amies de la classe \texttt{Polynome}. Définir ces 3 fonctions dans le fichier \texttt{polynome.cpp}.
    
    \item Tester toutes ces fonctions en utilisant le programme de test donné en exemple.
    
    \item Ajouter les opérateurs de comparaison \texttt{<} et \texttt{==}. L'ordre est donné par le degré des polynômes.
\end{enumerate}

\section*{Extensions}

L'idé maintenant est de mettre en évidence la souplesse de la POO. Par exemple, on peut décider de modifier la façon de stocker les diférents \texttt{Monome} dans le champ \texttt{data}. Un choix serait d'utiliser un tableau dynamique dont la taille n'est pas $n+1$ mais $k$ où $k$ est le nombre de monômes non nuls (par exemple 5 pour \texttt{P} et 1 pour \texttt{Q}). Un meilleur choix serait d'utiliser une liste chaînée de \texttt{Monome}.

\begin{enumerate}
    \setcounter{enumi}{13}
    \item Changer le champ \texttt{data} de la classe \texttt{Polynome} et modifier les méthodes et opérateurs de la classe \texttt{Polynome} de façon à  faire fonctionner le même programme de test.
\end{enumerate}

\end{document}



